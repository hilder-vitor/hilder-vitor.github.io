\documentclass[a4paper,11pt]{article}
\usepackage[utf8]{inputenc}
%
%\usepackage{epsfig}
\usepackage{amsfonts} % classic fonts (N, Q, R, C)
\usepackage{amsmath}  % equation, align*, \mod command
\usepackage{amsthm}   % \theoremstyle (to define remark and definition)
%\usepackage{amsfonts}
%\usepackage{amssymb}
%\usepackage{verbatim}

\usepackage[a4paper,hmargin=2.2cm,vmargin=2.7cm]{geometry} % 2cm, 2.3cm


%\usepackage{listings}
%\usepackage{algorithm,algorithmic}

% For algorithms
%\usepackage{algorithm2e}
\usepackage[linesnumbered,boxed,ruled,vlined]{algorithm2e}
%\usepackage{algpseudocode}

\usepackage{url}

%%%%%%%%%%%%%%%%%%%%%%%%%%%%%%%%%%%%%%%%%%%%%%%%%%%%%%%%%%%%%%%%%%%%%%%%%%%%%%%%
%%%%%    My own commands

\newcommand\N{{\mathbb N}}
\newcommand\Z{{\mathbb Z}}
\newcommand\Q{{\mathbb Q}}
\newcommand\R{{\mathbb R}}
\newcommand\Zq{{\mathbb Z}_q}
\newcommand\Zqn{{\mathbb Z}_q^n}
\newcommand\Zn{{\mathbb Z}^n}
\newcommand\Zqm{{\mathbb Z}_q^m}
\newcommand\Zm{{\mathbb Z}^m}
\newcommand\Zqnm{{\mathbb Z}_q^{n\times m}}
\newcommand\Znm{{\mathbb Z}^{n\times m}}

\newcommand\A{{\mathbf A}}
\newcommand\LqA{\mathcal{L}_q^{\bot}\left(\A\right)}
\newcommand\LB{\mathcal{L}\left(\vec B\right)}

\newcommand\DLs[1]{D_{\LB, s, #1}}

\newcommand\smooth{\eta_\epsilon}

%\newcommand\poly{\textsf{poly}}

\newcommand\SIS{\textbf{SIS}_{n, q, \beta, m}}
\newcommand\SVP{\textbf{searchSVP}}
\newcommand\gapSVP{\textbf{gapSVP}_\gamma}
\newcommand\SIVP{\textbf{SIVP}_\gamma}
\newcommand\incIVD{\textbf{IncIVD}_{\gamma, g}^{\smooth}}

\newcommand\sample{ \xleftarrow{\text{ \$ }} }

\DeclareMathOperator{\img}{\textsf{Img}}
\DeclareMathOperator{\poly}{\textsf{poly}}

\newcommand{\norm}[1]{\left\lVert #1 \right\rVert_2}

\let\vec\mathbf %%%% Redefining \vec to use bold letters instead of arrow


%%%%%%%%%%%%%%%%%%%%%%%%%%%%%%%%%%%%%%%%%%%%%%%%%%%%%%%%%%%%%%%%%%%%%%%%%%%%%%%%
%%%%% Theorems, corolaries, etc
 
\newtheorem{theorem}{Theorem}[section]
\newtheorem{corollary}{Corollary}[theorem]
\newtheorem{lemma}[theorem]{Lemma}
\theoremstyle{definition}
\newtheorem{definition}{Definition}[section]
\theoremstyle{remark}
\newtheorem*{remark}{Remark}
 
%%%%%%%%%%%%%%%%%%%%%%%%%%%%%%%%%%%%%%%%%%%%%%%%%%%%%%%%%%%%%%%%%%%%%%%%%%%%%%%%

% Title Page
\title{AKS Algorithm: Solving the exact SVP problem}
\author{Hilder Vítor Lima Pereira}
\date{June, 2019}



\begin{document}
\maketitle

%\begin{abstract}
%\end{abstract}

\section{Introduction}

Let $\vec{B}$ be a basis of an $n$-dimensional lattice $\mathcal{L}$.


Due to Ajtai, Kumar, and Sivakumar~\cite{aks2001}. 

Finds a shortest vector of a given lattice.

The running time of AKS is $2^{O(n)}\cdot \poly(n, B)$, where $B = max_{i, 
j}\log(|b_{i,j}|)$ for entries $b_{i,j}$ of $\vec B$. That is, a simple 
exponential in the dimension times a polynomial in the input size.

It is randomized, i.e., it samples random values during its execution and may 
output wrong answers, although it happens with negligible probability (the 
output is an actual shortest vector with probability exponentially close to 
one: $1 - 2^{-n}$).


\section{AKS algorithm for interval $[2, 3)$}

In this section, we show the AKS algorithm, which receives a lattice 
$\mathcal{L}$, assumes that $2 \le \lambda_1 < 3$, and returns a vector $\vec v 
\in \mathcal L$ such that $\norm{\vec v} = \lambda_1$ (with overwhelming 
probability).

In fact, if $\lambda_1 \not \in [2, 3)$, then AKS still returns a lattice 
vector, but then we do not have guarantees that it is a shortest non zero 
vector. Therefore, to solve the SVP for a general lattice (with $\lambda_1$ 
lying in any interval), we just transform a general lattice $\mathcal{L}$ in a 
sequence of lattices $\mathcal{L}_1$, $\mathcal{L}_2$, ..., $\mathcal{L}_\ell$ 
such that at least one $\mathcal{L}_k$ have a shortest nonzero vector inside 
the 
interval $[2, 3)$, and we run AKS for each of these lattices, returning the 
shortest nonzero vector found among the $\ell$ calls to AKS. We show it in more 
detail in the section~\ref{sec:from_2_3_to_general}.


\section{From interval $[2, 3)$ to any lattice}
\label{sec:from_2_3_to_general}

The goal of this section is to show that any algorithm to find a shortest 
nonzero vector in a lattice whose $\lambda_1 \in [2, 3)$ can be turned in an 
algorithm to find a shortest nonzero vector of general lattices.

Let $\mathcal{L}$ be an $n$-dimensional lattice and $\lambda_1$ be its first 
successive minima, i.e., the length of its shortest vector.

Using LLL, we can find an estimate $e$ for $\lambda_1$ such that
$$\lambda_1 \le e \le 2^n\lambda_1.$$

Manipulating that inequality, we get:
$$1 \le \frac{e}{\lambda_1} \le 2^n \Leftrightarrow \frac{1}{2^n} \le 
\frac{\lambda_1}{e} \le 1
\Leftrightarrow \frac{e}{2^n} \le \lambda_1 \le e.$$

Therefore, we know that the length of a shortest nonzero vector of 
$\mathcal{L}$ is in the interval $\left[ e\cdot 2^{-n}, e \right]$.

Now consider the lattice $\mathcal{L}' := \frac{2^{n+1}}{e}\mathcal{L}$. Then, 
clearly we have 
$$2 \le \lambda_1\left(\mathcal{L}'\right) \le 2^{n+1}.$$

Moreover, if $\vec v$ is a shortest nonzero vector of $\mathcal{L}'$, then 
$\frac{e}{2^{n+1}} \vec v$ is a shortest nonzero vector of $\mathcal{L}$. 
Therefore, it is sufficient to solve the SVP in $\mathcal{L}'$.

Furthermore, notice that by choosing $x = 3/2$, we have $3x^k = 2x^{k+1}$ for 
any $k \in \N^*$, and then the interval in which 
$\lambda_1\left(\mathcal{L}'\right)$ lies can be partitioned in intervals of 
the form $[2x^k, 3x^k)$, that is, for some $\ell \in \N$, we have:

$$[2, 2^{n+1}] \subset [2, 3) \cup [2x, 3x) \cup [2x^2, 3x^2) \cup ... \cup 
[2x^\ell, 3x^\ell).$$

We need an $\ell$ such that $3x^\ell > 2^{n+1} \Leftrightarrow 3(3/2)^\ell > 
2^{n+1} \Leftrightarrow 3^{\ell+1} > 2^{\ell + n+1}$, an it is 
sufficient to take $\ell = 2n$.

Thus, we can assume that $\lambda_1(\mathcal{L}')$ lies in some of those 
intervals $[2x^k, 3x^k)$. But in this case, $x^{-k}\lambda_1(\mathcal{L}') \in 
[2, 3)$. Therefore, defining $\mathcal{L}_k := x^{-k}\mathcal{L}'$, we can use 
the algorithm that solves SVP in $[2, 3)$ to find a shortest vector $\vec v \in 
\mathcal{L}_k$. Then, $x^k\vec v$ is a shortest vector of $\mathcal{L}'$, and 
finally, $\left(\frac{e}{2^{n+1}}\right)x^k\vec v$ is a shortest vector of the 
original lattice $\mathcal{L}$.

Hence, if we have an algorithm $\mathcal{A}$ that solves the SVP when the first 
minima belongs to $[2, 3)$, then given a lattice $\mathcal{L}$, we can define 
$\mathcal{L}'$ as above and for each $k \in \{0, 1, ..., 2n\}$, define 
$\mathcal{L}_k := x^{-k}\mathcal{L}$ and $\vec{v}_k := 
\mathcal{A}\left(\mathcal{L}_k\right)$. Then, set $\vec v$ as the shortest 
vector 
among those $\vec{v}_0, \vec{v}_1, ..., \vec{v}_{2n}$ and output 
$\left(\frac{e}{2^{n+1}}\right)x^k\vec v$, which will be a shortest vector of 
$\mathcal{L}$.

Notice that, denoting by $T(n)$ the time complexity of $\mathcal{A}$, then we 
can solve SVP using such method in time $O(n \cdot T(n))$.


\begin{definition}[Shortest Vector Problem: $\SVP$] Given a vector $\LB$, find 
a vector $\vec v \in \LB$ such that $\norm{\vec v} = \lambda_1(\LB)$.
\end{definition}

\begin{lemma}
Testing Lemma
\end{lemma}

\begin{proof}
It is trivial.
\end{proof}


\begin{remark}
Testing remark
\end{remark}

\begin{theorem}[Testing theorem]
A fake but nice theorem.
\end{theorem}
\begin{proof}
It is also trivial.
\end{proof}

\begin{algorithm}
\DontPrintSemicolon
\KwIn{A basis $\vec B$ of a lattice $\mathcal{L}$}
\KwOut{A vector $\vec v \in \LB$ that is a solution to $\SVP$.}
\SetKwComment{Comment}{$\triangleright$\ }{}

$j \sample \{1,...,m\}$

\For{$\ell = 1$ \emph{ until } $2n$}{
    \lIf{$\ell = j$}{
		$\vec y_i \gets \DLs{\vec c_j}$
    }
    \lElse{
		$\vec y_i \gets \DLs{\vec 0}$
    }
}

$\vec A := \vec{B}^{-1} \vec Y \mod q \in \Zqnm$ \Comment*[r]{The oracle 
$\mathcal{O}$ expects 
$\A$ to be uniform in $\Zqnm$.}

$\vec e \gets \mathcal{O}(\A)$ \Comment*[r]{$\vec e$ should satisfy $\norm{\vec 
e} \le \beta$ and $\A\vec e = \vec 0 \mod q$.}

\Return{$\vec v$}
\caption{{\sc AKS}}
\label{alg:AKS}
\end{algorithm}

The first important thing to do in order to analyze 
algorithm~\ref{alg:AKS} is...

\bibliographystyle{alpha}
\bibliography{aks.bib}

\end{document}


