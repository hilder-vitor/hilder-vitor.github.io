\documentclass[aspectratio=43]{beamer}

\usepackage[utf8]{inputenc}
\usepackage[english]{babel}
\usepackage{csquotes}%Correctly typeset Vojtech
\usepackage{csvsimple}
\usepackage{booktabs}
\usepackage{styles/beamer_eit-en}
\usepackage[font={scriptsize}]{caption}%Smaller image captions, it for italics, can choose footnotesize to be even smaller
\captionsetup[figure]{labelformat=empty}%Remove "Figure" from the figure caption
%\setbeamertemplate{footline}[frame number]
%\setbeamerfont{footline}{series=\bfseries}

\usepackage{tikz}
\usepackage{subcaption}

% Algorithms packages
\usepackage[linesnumbered,boxed,ruled,vlined]{algorithm2e}
%\usepackage{algorithm2e}
%\usepackage{algorithm,algpseudocode}
%\usepackage{algorithm,algorithmic}

%Mathematic packages
\usepackage{amsmath}
\usepackage{amsfonts}
\usepackage{amsbsy}

%\usepackage[backend=biber,
%style=authoryear,]{biblatex}
\usepackage{biblatex}

% File is created and written to disk by the above package
%\addbibresource{literature.bib}


\newcommand{\norm}[1]{\left\lVert #1 \right\rVert_2}
\newcommand{\ball}[1]{\mathcal{B}(0, #1)}
\DeclareMathOperator{\poly}{\textsf{poly}}
\DeclareMathOperator{\vol}{\textsf{vol}}

\let\vec\mathbf %%%% Redefining \vec to use bold letters instead of arrow


%%%%%%%%%%%%%%%%%%%%%%%%%%%%%%%%%%%%%%%%%%%%%%%%%%%%%%%%%%%%%%%%%%%%%%%%%%%%%%%%
%%%%% Theorems, corolaries, etc

%\newtheorem{theorem}{Theorem}[section]
%\newtheorem{corollary}{Corollary}[theorem]
%\newtheorem{lemma}[theorem]{Lemma}
%\theoremstyle{definition}
%\newtheorem{definition}{Definition}[section]
\theoremstyle{remark}
\newtheorem*{remark}{Remark}

\title[AKS Algorithm]{AKS Algorithm: Finding shortest nonzero vectors.}

\author{Hilder Vitor Lima Pereira}

\institute{\emph{Introduction to lattices and their applications in Computer 
Science/Cryptography.}\\
	Doctoral Programme in Computer Science and Computer Engineering\\
	University of Luxemburg.\\
}
\date{12th June, 2019}

\begin{document}



\begin{frame}
        \maketitle
\end{frame}

\begin{frame}{Table of contents}
        \tableofcontents
\end{frame}

\section{Introduction}

\begin{frame}
\frametitle{Introduction}

\textbf{SVP:}

Given a lattice $\mathcal{L}$, the shortest nonzero vector problem (SVP) is 
the problem of finding a point $\vec v \in \mathcal{L} \setminus \{\vec{0}\}$ 
such that the Euclidean norm is minimized, that is, $\forall \vec{u} \in 
\mathcal{L}\setminus \{ \vec 0\}, \norm{\vec{v}} \le \norm{\vec{u}}$.
\end{frame}

\begin{frame}
\frametitle{Introduction}

\textbf{Solving SVP:}
\vspace*{0.25cm}

Algorithms for approximate versions of SVP.
\begin{itemize}
	\item LLL solves approx. SVP with  exponential approximation factor in 
	polynomial-time.
	\item BKZ solves approx. SVP with polynomial approximation factor roughly 
	in exponential time.
\end{itemize}
\pause 
\vspace*{0.2cm}
\textbf{What about AKS?}
\vspace*{0.25cm}
\begin{itemize}
	\item It solves the exact version of SVP.
	\item Exponential time and memory.
	\item It is a randomized algorithm.
	\item It outputs the correct answer with overwhelming probability.
\end{itemize}
\end{frame}


\begin{frame}
\frametitle{Overview}

\begin{enumerate}
\item Sample $2^{O(n)}$ random lattice points inside a ball 
$\ball{R}$.

\item Find ``centers points'' among them, i.e., points close to several other 
points.

\item Get new lattice points by computing the difference between the points and 
their centers. (Note that the new points lie in $\ball{R'}$ with $R' < R$).

\item Repeat it with those new points unless they are already shorter than some 
bound.

\item Output the shortest vector among the remaining ones.
\end{enumerate}
\end{frame}

\begin{frame}
\frametitle{1. How does one sample random lattice points within a ball?}

We actually sample them indirectly:
\begin{itemize}
\item Sample a \emph{real} vector $\vec x \in \ball{R} \cap \mathbb{R}^n$.
\item Compute $\vec y \in \mathcal{P}(\vec B)$ (the fundamental region)  such 
that $\vec y - \vec x \in \mathcal{L}$.
\item Define $\vec z = \vec y - \vec x$ as the random lattice point.
\end{itemize}

\vspace*{0.25cm}
We compute $\vec y$ as $\vec x \mod \vec B :=  \mathbf{x} - \mathbf{B} 
\lfloor\mathbf{B}^{-1} \mathbf{x} \rfloor$, which is a ``reduction modulo the 
basis $\vec B$''.

\vspace*{0.25cm}
Notice that $\vec z$ is the corner of the translated fundamental region that 
covers $\vec x$.

\end{frame}


\begin{frame}[fragile]
\frametitle{2. How does one find center points?}

We use a procedure known as Sieve:

\begin{algorithm}[H]
%	\KwResult{Write here the result }
%	initialization\;
\KwIn{A positive $R \in \mathbb{R}$ and $X := \{\vec{x}_1, ..., 
\vec{x}_N\}\subset\ball{R}$}
\KwOut{A set $C$ of pairs $(i, j)$ such that $\vec{x}_j$ is the center of 
$\vec{x}_i$.}
\SetKwComment{Comment}{$\triangleright$\ }{}
$C := \emptyset$

\For{$i = 1$ \emph{ until } $N$}{
	
	\If{$\exists (i', j) \in C$ such that $\norm{\vec{x}_i - \vec{x}_j} \le 
	R/2$}{
		$C = C \cup \{(i, j)\}$ \Comment*[r]{$\vec{x}_j$ becomes the center of 
		$\vec{x}_i$}
	}\Else{
		$C = C \cup \{(i, i)\}$ \Comment*[r]{$\vec{x}_i$ becomes its own center}
	}
}
\caption{{\sc SIEVE}}
\end{algorithm}
\vspace*{0.25cm}
Notice that the ``centers'' are defined by the second entry of the pairs $(i, 
j)$ in $C$. For each $\vec{x}_i$, its center is $\vec{x}_j$.

\end{frame}


\begin{frame}
\frametitle{2. How does one find center points?}

\begin{lemma}
Let $R \in \mathbb{R}_{>0}$. For any set of points $X = \{\vec{x}_1, ..., 
\vec{x}_N \} \subset \ball{R}$, let $C$ be the set returned by {\sc SIEVE}. Then
\begin{itemize}
\item[(i)] $C$ defines at most $5^n$ centers and
\item[(ii)] $\forall (i, j) \in C$, $\norm{\vec{x}_i - \vec{x}_j} \le R/2$
\end{itemize}
Moreover, (iii) {\sc SIEVE} runs in polynomial time in the input size.
\end{lemma}


\end{frame}


\begin{frame}
\frametitle{2. How does one find center points?}

\begin{proof}
Define balls of radius $\frac{R}{4}$ around each center.
Notice that they are disjoint, because the distance between two centers is 
bigger than $\frac{R}{2}$.

Furthermore, their union is contained in 
$\ball{\frac{5R}{4}}$.

Therefore, the number of balls (which equals the number of centers) is at most

$$\frac{\vol(\ball{\frac{5R}{4}})}{\vol(\ball{\frac{R}{4}})} = 5^n.$$

Propositions (ii) and (iii) are trivial.
\end{proof}

\vspace*{0.5cm}
Remember that $\vol(\ball{R}) = \frac{\pi^{n/2}R^n}{\Gamma(n/2+1)}$.

\end{frame}

\section{AKS for $\lambda_1 \in [2, 3)$}

\begin{frame}
\frametitle{AKS for $\lambda_1 \in [2, 3)$}

We are almost ready to see the algorithm AKS. Before defining it to the general 
case, let's assume we are working over lattices for which

$$\lambda_1 \in [2, 3).$$

We will see in the end how to remove this restriction.
\end{frame}

\begin{frame}[fragile]
\begin{algorithm}[H]
	%	\KwResult{Write here the result }
	%	initialization\;
	\KwIn{A basis $\vec B$ of an $n$-dimensional lattice whose 
		$\lambda_1 \in [2, 3)$}
	\KwOut{A shortest nonzero vector of $\mathcal{L}(\vec B)$}
	\SetKwComment{Comment}{$\triangleright$\ }{}
	$R := n\cdot \max \norm{\vec{b}_j} + 2$

	$N := 2^{8n}\log{R}$

	Sample $X:=\{\vec{x}_1, ..., \vec{x}_N \}$ unif. in $\ball 2 \cap 
	\mathbb{R}^n$
	
	$Y := \{\vec{y}_i := \vec{x}_i \mod \vec{B} : \vec{x}_i \in X \} \subset 
	\mathcal{P}(\vec B)$
	
	\While{$R > 6$}{
		$C := \text{{\sc SIEVE}}(Y)$
		
		\For{each center $\vec{y}_j$ defined by $C$}{
			$Y = Y \setminus \{\vec{y}_j\}$; $X = X \setminus \{\vec{x}_j\}$
		}
		
		\For{each $\vec{y}_j$ in $Y$}{
			Let $\vec{y}_c$ be the center of $\vec{y}_j$
			
			$\vec{y}_j = \vec{y}_j - (\vec{y}_c - \vec{x}_c)$
		}
		$R = R/2 + 2$
	}
	Return the shortest $(\vec{y}_i - \vec{x}_i) - (\vec{y}_j - \vec{x}_j)$ 
	(among $Y$ and $X$)
	\caption{{\sc AKS*}, for $\lambda_1 \in [2, 3)$}
\end{algorithm}
\end{frame}


\begin{frame}
\frametitle{AKS for $\lambda_1 \in [2, 3)$}

\begin{lemma}
The number of iterations of the while loop in {\sc AKS*} is at most $2\log 
R_0$, where $R_0$ is the first value assigned to $R$.
\end{lemma}

\pause
\begin{proof}
Let $R_k$ be the value of $R$ on the beginning of the $k$-th iteration.

Then, we have $R_1 = R_0$, $R_2 = R_0/2 + 2$, $R_3 = R_0/2^2 + 1 + 2$, $R_4 = 
R_0/2^3 + 1/2 + 1 + 2$, etc.

In general, $R_k = R_0 / 2^{k-1} + 2 + \sum_{i=0}^{k-3}1/2^i$.

For $k = \lceil \log R_0 \rceil + 1$ we have $R_k \le 1 + 2 + 2 = 5 < 6$ and 
then the while loop is aborted.

Therefore, the number of iterations is at most 
$$\lceil \log R_0 \rceil + 1 \le 2\log R_0.$$
\end{proof}

\end{frame}


\begin{frame}
\frametitle{AKS for $\lambda_1 \in [2, 3)$}

\begin{lemma}
{\sc AKS*} runs in time $2^{O(n)}$ times some polynomial in the
input size.
\end{lemma}

\pause
\begin{proof}
	
	Let $S = \log(R_0)$ be the input length.
	\vspace*{0.1cm}
	
	The initialization step, before the while loop, already costs 
	$2^{O(n)}\poly(S)$. The final step, after the loop, is clearly cheaper than 
	this (since we have removed several points from $Y$ and $X$).
	\vspace*{0.1cm}	
	
	The procedure {\sc SIEVE} runs in polynomial time in the number of points 
	it receives, that is, $O(\poly(2^{8n}\log R)) = 2^{O(n)}\poly(S)$.
	\vspace*{0.1cm}
	
	By the last lemma, {\sc SIEVE} is executed at most $2S$ times,
	therefore, the cost of the loop is also $2^{O(n)}\poly(S)$.
\end{proof}

\end{frame}


\begin{frame}
\frametitle{AKS for $\lambda_1 \in [2, 3)$}

\begin{lemma}
Let $Z := \{(\vec{x}_i, \vec{y}_i) : \vec{x}_i \in X \land \vec{y}_i := 
\vec{x}_i \mod \vec B \}$.
At the end of {\sc AKS*}, the set $Z$ has an exponential number of pairs 
and each pair gives us a lattice vector with norm bounded by $8$.
\end{lemma}

\begin{remark}
Several pairs $(\vec{x}_i, \vec{y}_i)$ and $(\vec{x}_j, \vec{y}_j)$ define the 
same lattice point...
\end{remark}

\end{frame}


\begin{frame}
\frametitle{AKS for $\lambda_1 \in [2, 3)$}

\begin{proof}
	
	By the definition of $\vec{y}_i$, at the beginning of the algorithm, we 
	have $\vec{y}_i \in \mathcal{P}(\vec B)$, thus, $\norm{\vec{y}_i} \le 
	\sum\norm{\vec{b}_j} \le R_0$. And at each iteration $k$, $\vec{y}_i$ is 
	updated 
	to $\vec{y}_i - (\vec{y}_c - \vec{x}_c)$, therefore, its norm becomes
	$$\norm{\vec{y}_i - (\vec{y}_c - \vec{x}_c)} \le \norm{\vec{y}_i - 
		\vec{y}_c} + \norm{ \vec{x}_c} \le R_k/2 + 2.$$
	
	Thus, at the end of the last iteration, we have $\norm{\vec{y}_i} \le 6$.
	
	Therefore, we have $\vec{y}_i - \vec{x}_i \in \mathcal{L}$ and 
	$$\norm{\vec{y}_i - \vec{x}_i} \le 6 + 2 = 8.$$
	
	Now notice that at the each iteration, at most $5^n$ points are removed 
	from $X$ and $Y$, therefore 
	$$|Z| \ge N - 5^n\cdot 2\log R_0 = (2^{8n} - 2\cdot 
	5^n)\log R_0 \ge 2^{7n}\log R_0.$$
\end{proof}

\end{frame}


\begin{frame}
\frametitle{AKS for $\lambda_1 \in [2, 3)$}

Lets breath a bit...
\begin{itemize}
\item We have proved that {\sc AKS*} finds an exponentially large 
set of pairs which define (possibly repeated) very short lattice points.

\item Remember that we are supposing $\lambda_1 \in [2, 3)$ and all those 
lattice points have norm smaller than $8$. Therefore, they are already a very 
good approximation to a shortest nonzero vector.

\item Intuitively, it is very likely that a shortest 
nonzero vector is indeed among them.
\end{itemize}
\end{frame}


\begin{frame}
\frametitle{AKS for $\lambda_1 \in [2, 3)$}

How can we prove that {\sc AKS*} really finds a shortest nonzero vector with 
high probability?

\vspace*{0.5cm}

\emph{Intuition:}

\begin{itemize}
\item Notice that if we sample the points $\vec{x}_i$ differently, but keeping 
the same distribution, the algorithm's output must be the same.

\item For analysis purposes, sample $\vec{x}_i$ such that many of them are 
equal to a fixed $\vec{w}$ and many have the form $\vec{w} \pm \vec{v}$, where  
$\vec{v}$ is a shortest nonzero vector.

\item Then, at the end of the algorithm, with high probability, we will have 
$\vec{x}_i$ and $\vec{x}_j$ whose difference equals $\pm\vec{v}$.

\end{itemize}
\end{frame}


\begin{frame}
\frametitle{AKS for $\lambda_1 \in [2, 3)$}

A lemma that we will need later...

\begin{lemma}
Let $\mathcal{L}$ be a lattice such that $\lambda_1 \in [2, 3)$. Then there are 
at most $9^{n}$ lattice points inside $\ball{8}$.
\end{lemma}

\pause

\begin{proof}
Let $m$ be the number of points in $\mathcal{L} \cap \ball{8}$.

Because $\lambda_1 \ge 2$, we can consider $m$ disjoint balls of radius $1$ 
centered in 
each lattice point inside $\ball{8}$.

Then, the union of all these balls is contained in $\ball{9}$.

Thus, we have $m \cdot \vol(\ball{1}) \le \vol(\ball{9})$. Therefore,

$$m \le \frac{\vol(\ball{9})}{\vol(\ball{1})} = 9^n.$$


\end{proof}

\end{frame}
\begin{frame}
\frametitle{AKS for $\lambda_1 \in [2, 3)$}

\begin{theorem}
If $\lambda_1 \in [2, 3)$, then {\sc AKS*} returns a shortest nonzero vector 
with probability exponentially close to 1, i.e., bigger than $1 - 2^{-n}$. 
\end{theorem}
\end{frame}


\begin{frame}
\frametitle{AKS for $\lambda_1 \in [2, 3)$}

Sketch of the proof
\vspace*{0.25cm}

Let $\vec{v}$ be a shortest nonzero vector, thus $\norm{\vec{v}} \in [2, 3)$.

Define $C_1 := \ball{2}\cap \mathcal{B}(-\vec{v}, 2)$ and $C_2 := \ball{2}\cap 
\mathcal{B}(\vec{v}, 2)$.
\pause

\begin{figure}
\centering
\begin{subfigure}[b]{0.45\textwidth}
\begin{tikzpicture}[scale=0.50, transform shape]
\begin{scope}
\clip (1, 3) circle (2);
\clip (3, 3) circle (2);
\fill[color=blue!20] (-2, 0) rectangle (10, 6);
\end{scope}
\begin{scope}
\clip (3, 3) circle (2);
\clip (5, 3) circle (2);
\fill[color=blue!20] (-2, 0) rectangle (10, 6);
\end{scope}
\draw (1, 3) circle (2);
\draw (3, 3) circle (2);
\draw (5, 3) circle (2);
\filldraw[black] (1,3) circle (2pt) node[anchor=north] {$-\vec v$};
\filldraw[black] (3,3) circle (2pt) node[anchor=north] {(0,0)};
\filldraw[black] (5,3) circle (2pt) node[anchor=north] {$\vec v$};
\node at (2, 3.5) {$C_1$};
\node at (4, 3.5) {$C_2$};
\end{tikzpicture}
\caption{Example for $\norm{\vec v} = 2$.}
\end{subfigure}
\quad
\begin{subfigure}[b]{0.45\textwidth}
\begin{tikzpicture}[scale=0.50, transform shape]
\begin{scope}
\clip (0, 3) circle (2);
\clip (3, 3) circle (2);
\fill[color=blue!20] (-2, 0) rectangle (10, 6);
\end{scope}
\begin{scope}
\clip (3, 3) circle (2);
\clip (6, 3) circle (2);
\fill[color=blue!20] (-2, 0) rectangle (10, 6);
\end{scope}
\draw (0, 3) circle (2);
\draw (3, 3) circle (2);
\draw (6, 3) circle (2);
\filldraw[black] (0,3) circle (2pt) node[anchor=north] {$-\vec v$};
\filldraw[black] (3,3) circle (2pt) node[anchor=north] {(0,0)};
\filldraw[black] (6,3) circle (2pt) node[anchor=north] {$\vec v$};
\node at (1.5, 3.3) {$C_1$};
\node at (4.5, 3.3) {$C_2$};
\end{tikzpicture}
\caption{Example for $\norm{\vec v} = 3$.}
\end{subfigure}
%\caption{Figure caption here}
\end{figure}
\end{frame}


\begin{frame}
\frametitle{AKS for $\lambda_1 \in [2, 3)$}

Sketch of the proof
\vspace*{0.25cm}

Define the function $\tau: X \to X$ that flips vectors from $C_1$ to $C_2$ and 
vice-versa:

\begin{equation*}
\tau(\vec{x}_i) = \begin{cases}
               \vec{x}_i + \vec{v},    & \text{if } \vec{x}_i \in C_1\\
               \vec{x}_i - \vec{v},    & \text{if } \vec{x}_i \in C_2\\
               \vec{x}_i,              & \text{otherwise}
           \end{cases}
\end{equation*}

Notice that $\tau$ is a bijection, therefore, $X$ and $\tau(X)$ follow the same 
distribution.

Moreover, $\vec{x}_i = \vec{x}_i \pm \vec{v} \mod \vec B$, therefore 
$$\vec{y}_i = \vec{x}_i \mod \vec B \Leftrightarrow \vec{y}_i = \tau(\vec{x}_i) 
\mod \vec B.$$

Therefore, {\sc AKS*} has the same output given $X$ or $\tau(X)$.
\end{frame}


\begin{frame}
\frametitle{AKS for $\lambda_1 \in [2, 3)$}

Sketch of the proof
\vspace*{0.25cm}

Hence, apply $\tau$ to all $\vec{x}_i$.

As proved earlier, we have more than $2^{7n}$ vectors $\vec{x}_i$ at the end of 
the algorithm. For each of them, we have a lattice point $\vec{z}_i := 
\vec{y}_i - \vec{x}_i$ and $\vec{z}_i \in \ball{8}$. But there are at most 
$9^{n}$ lattice points inside $\ball{8}$.

Therefore, there exists $\vec{w} \in \mathcal{L}$ yielded by at least $2^{7n} / 
9^n \ge 2^{3.8n}$ pairs of $\vec{x}_i$ and $\vec{y}_i$.

Then, with high probability*, for such $\vec{w}$, at least one $\vec{x}_i$ is 
in 
$C_1 \cup C_2$ and at least one $\vec{x}_j$ doesn't belong to $C_1 \cup C_2$.

But then, $\vec{x}_i$ is flipped by $\tau$ and $\vec{x}_j$ isn't, and 
therefore, {\sc AKS*} returns 
$$(\vec{y}_i - \vec{x}_i \pm \vec{v}) - (\vec{y}_j - \vec{x}_j) = \vec{w} \pm 
\vec{v} - \vec{w} = \pm \vec{v}.$$



\end{frame}


\begin{frame}
Sketch of the proof
\vspace*{0.25cm}

* It is not hard to see that for such $\vec{w}$, at least one $\vec{x}_i$ is in 
$C_1 \cup C_2$ with probability bigger than $1 - 2^{-n}$.

Notice that $\mathcal{B}(\vec{v}/2, 0.5)$ fits in $C_1$ and also in $C_2$. 
Hence, $\vol(C_1) \ge \vol(\mathcal{B}(\vec{v}/2, 0.5))$. Therefore, 

$$\frac{\vol(C_1)}{\vol(\ball{2})} \ge \frac{\vol(\mathcal{B}(\vec{v}/2, 
0.5))}{\vol(\ball{2})} = \frac{0.5^n}{2^n} = 2^{-2n}.$$

Thus, we have $\forall \vec{x}_i \leftarrow \mathcal{U}\left(\ball{2}\right), 
Pr[\vec{x}_i \in C_1 \cup C_2] \ge 2\cdot 2^{-2n}$.

Then, considering the (more than) $2^{3.8n}$ vectors $\vec{x}_i$ associated to 
$\vec{w}$ and using Chebyshev's inequality, we have 
%$$Pr[\forall i, \vec{x}_i \not \in C_1 \cup C_2] \le \left(1 - 2\cdot 
%2^{-2n}\right)^{2^{3.8n}}$$
%and therefore,
$$Pr[\exists i : \vec{x}_i \in C_1 \cup C_2] \ge 1 - 
\frac{2^{2n}}{2^{3.8n}} = 1 - \frac{1}{2^{1.8n}} \ge 1 - 2^{-n}.$$

\end{frame}


\begin{frame}
\frametitle{AKS for $\lambda_1 \in [2, 3)$}

\begin{proof}[Sketch of the proof]
* Using a similar argument we can also see that for such $\vec{w}$, at least 
one $\vec{x}_i$ is outside $C_1 \cup C_2$ with probability bigger than $1 - 
2^{-n}$.
\end{proof}
\end{frame}


\section{AKS for general lattices}
\begin{frame}
\frametitle{Removing the restriction $\lambda_1 \in [2, 3)$}

``Okay, nice. But our lattices don't have such a small $\lambda_1$!''

\hfill - Impatient audience.

\end{frame}

\begin{frame}
\frametitle{Removing the restriction $\lambda_1 \in [2, 3)$}

Using LLL, we can find an estimate $e$ for $\lambda_1$ such that
$$\lambda_1 \le e \le 2^n\lambda_1.$$

Manipulating that inequality, we get:
$$1 \le \frac{e}{\lambda_1} \le 2^n \Leftrightarrow \frac{1}{2^n} \le 
\frac{\lambda_1}{e} \le 1
\Leftrightarrow \frac{e}{2^n} \le \lambda_1 \le e.$$

Therefore, we know that the length of a shortest nonzero vector of 
$\mathcal{L}$ is in the interval $\left[ \frac{e}{2^{n}}, e \right]$.

\end{frame}


\begin{frame}
\frametitle{Removing the restriction $\lambda_1 \in [2, 3)$}

\begin{itemize}
\item Consider the lattice $\mathcal{L}' := \frac{2^{n+1}}{e}\mathcal{L}$.

\item Then $2 \le \lambda_1\left(\mathcal{L}'\right) \le 2^{n+1}.$

\item If $\vec v$ is a shortest nonzero vector of $\mathcal{L}'$ , then 
$\frac{e}{2^{n+1}} \vec v$ is a shortest nonzero vector of $\mathcal{L}$. 

\item Therefore, it is sufficient to solve the SVP on $\mathcal{L}'$.

\end{itemize}

\end{frame}


\begin{frame}
\frametitle{Removing the restriction $\lambda_1 \in [2, 3)$}

How to solve SVP on $\mathcal{L}'$ knowing that $\lambda_1(\mathcal{L}') \in 
[2, 2^{n+1}]$?

\begin{itemize}
\item Write
$$[2, 2^{n+1}] \subset [2, 3) \cup [2x, 3x) \cup [2x^2, 3x^2) \cup ... \cup 
[2x^\ell, 3x^\ell).$$

\item By choosing $x = 3/2$, we have $3x^k = 2x^{k+1}$.

\item We need an $\ell$ such that $3x^\ell > 2^{n+1} \Leftrightarrow 
3(3/2)^\ell > 2^{n+1} \Leftrightarrow 3^{\ell+1} > 2^{\ell + n+1}$, an it is 
sufficient to take $\ell = 2n$.

\item Then, for some $k \in \{0, ..., \ell\}$, $\lambda_1(\mathcal{L}') \in 
[2x^k, 3x^k)$.

\item Therefore, for such $k$, $\lambda_1(x^{-k}\mathcal{L}') = 
x^{-k}\lambda_1(\mathcal{L}') \in [2, 3)$.
\end{itemize}
\end{frame}

\begin{frame}[fragile]
\frametitle{The AKS algorithm (for any $\lambda_1$)}
\begin{algorithm}[H]
	%	\KwResult{Write here the result }
	%	initialization\;
	\KwIn{A basis $\vec B$ of an $n$-dimensional lattice $\mathcal{L}$}
	\KwOut{A shortest nonzero vector of $\mathcal{L}(\vec B)$}
	\SetKwComment{Comment}{$\triangleright$\ }{}
	Run LLL to get an estimate $e$ for $\lambda_1$

	Define $\mathcal{L}' := \frac{2^{n+1}}{e}\mathcal{L}$ \Comment*[r]{Just 
	multiply $\vec{B}$ by $\frac{2^{n+1}}{e}$}

	\For{$k = 0$ until $2n$}{
		Define $\mathcal{L}_k := x^{-k}\mathcal{L}'$

		$\vec{v}_k = $ {\sc AKS*}($\mathcal{L}_k$)
	}
	
	Let $\vec{v}$ be a shortest nonzero vector among all $\vec{v}_k$
	
	Let $\vec{u} = x^k\vec{v}$ be a shortest nonzero vector of $\mathcal{L}'$
	
	Return $\frac{e}{2^{n+1}}\vec{u}$
	\caption{{\sc AKS}}
\end{algorithm}
\end{frame}

\begin{frame}
\frametitle{References}
\footnotesize{
	\begin{thebibliography}{99} % Beamer does not support BibTeX so references 
	%must be nserted manually as below
		\bibitem[AKS2001]{p1} M. Ajtai, R. Kumar, D. Sivakumar (2001)
		\newblock A sieve algorithm for the shortest vector problem.
		\newblock \emph{Procedings of the thirty-third annual ACM symposium on 
		Theory of 
		Computing}. Pages 601 - 610.
	\end{thebibliography}
	
	
	\begin{thebibliography}{99} % Beamer does not support BibTeX so 
		%references 
		%must be nserted manually as below
	\bibitem[Regev2004]{p1} Oded Regev (2004)
	\newblock $2^{O(n)}$-time algorithm for SVP.
	\newblock \emph{Lecture notes: Lattices in Computer Science}. Tel Aviv 
	University.
	\end{thebibliography}
	
}
\end{frame}


\end{document}
